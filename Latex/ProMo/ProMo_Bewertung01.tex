\documentclass[11pt]{article}
\usepackage{a4wide}
\usepackage{stmaryrd}
\usepackage{amsmath}
\usepackage{amssymb}
\usepackage{ngerman}
\usepackage{bussproofs}
\usepackage[all]{xy}
\usepackage[utf8]{inputenc}
\usepackage{graphicx}
% \usepackage{fancyvrb}
\usepackage{alltt}
\usepackage{upquote} % allows backticks and ticks to be copied from verbatim into editors without changing them
\usepackage{enumerate}
\usepackage{hyperref}
\usepackage{../common}
\usepackage{../mymath}
\usepackage{../mylogik}

\newcommand{\limp}{\to}
\newcommand{\liff}{\leftrightarrow}

\pagestyle{empty}


%%%%%%%%%%%%%%%%%%%%%
    % \noloesung
    % \nobewertung
%%%%%%%%%%%%%%%%%%%%%

\begin{document}
\uheader{1}{21. April 2015}


\begin{aufgabe}{Kartesisches Produkt}\label{kartesischesProdukt}
Berechnen Sie mit Papier und Bleistift die Menge des jeweiligen
kartesischen Produktes. 
Wenn Sie möchten, können Sie ihre Antworten mit GHCI überprüfen.

\emph{Hinweis:} im Gegensatz zu Listen ist 
      die Reihenfolge der Elemente einer Menge
      unerheblich (und Mengen enthalten auch keine ``doppelten'' Elemente).
\begin{enumerate}
 \item $\{11, 7\} \times \{\clubsuit, \heartsuit\}$
      \Loesung{$\bigl\{(11,\clubsuit),(11,\heartsuit),(7,\clubsuit),(7,\heartsuit)\bigr\}$ }

%  \item Was sind die Elemente des Produkt $\{1, 2\} \times \{3, 4, 5\}$?
%     \Loesung{$\{(1,3),(1,4),(1,5),(2,3),(2,4),(2,5)\}$}

  \item $(\{1, 2\} \times \{3, 4\}) \times \{5,6\}$
    \Loesung{$\bigl\{(1,3,5),(1,3,6),(1,4,5),(1,4,6),(2,3,5),(2,3,6),(2,4,5),(2,4,6)\bigr\}$}

  \item $\{1, 2\} \times (\{3, 4\} \times \{5,6\})$
    \begin{loesung}
      Abgesehen von der inneren Klammerung ist
      das Ergebnis identisch zur vorherigen Teilaufgabe.
      In der Mathematik ist es oft üblich, nicht zwischen diesen beiden Produkten zu unterscheiden und 
      gleich Tripel zu verwenden,
      d.h.\ obwohl das kartesische Produkt eigentlich nicht assoziativ ist, wird es oft als assoziativ behandelt.
      Funktionale Sprachen unterscheiden jedoch üblicherweise zwischen solchen isomorphen Produkten,
      denn \verb|(1,(2,3))| hat einen anderen Typ als \verb|((1,2),3)| oder \verb|(1,2,3)|.
    \end{loesung}
  
  \item $\{1, 2\} \times \{\}$
    \Loesung{Das Produkt ist hier die leere Menge.}
    
  \item Sei $A$ eine Menge mit $n$ Elementen. Wie viele Elemente gibt es in $A \times \{27,69\}$?
    \Loesung{Doppelt so viele, da jedes Element aus $A$ zwei Mal im Produkt mit $\{27,69\}$ vorkommt; also insgesamt $2n$.}
  
\end{enumerate}
\end{aufgabe}
% 
% \begin{aufgabe}{Grundlegende Operatoren}\label{BasicOps}
% % \begin{enumerate}
% %   \item Implementieren Sie eine Funktion, welche zwei Argumente des Typs \verb|Bool|
% %     zu einem Ergebnis des Typs \verb|Bool| verarbeitet. Ihre Funktion soll nur 
% %     genau dann den Wert \verb|True| liefern, wenn beide Argumente \verb|True| sind.
% %     \begin{loesung}
% %       \begin{verbatim}
% %         myAnd :: Bool -> Bool -> Bool
% %         myAnd True True = True
% %         myAnd _    _    = False
% %       \end{verbatim}
% %     \end{loesung}
% 
% %   \item  
%     Die Funktionen für die logischen Operatoren UND, ODER und Negation 
%     wurden nicht in der Vorlesung erklärt.
%     Finden Sie selbst heraus, wie diese Operatoren in Haskell heißen 
%     und testen Sie deren Funktionsweise kurz mit GHCI.
%     
%     \emph{Hinweise:} 
%       Die Vorlesungshomepage listet Links zur Lösung dieser Aufgabe.      
%       Zum Beispiel können Sie in der Dokumentation der Standardbibliotheken nachschlagen;
%       grundlegende Funktionen findet man oft im Modul \verb!Prelude! oder
%       auch in den \verb|Data|-Modulen, z.B.\ \verb!Data.Bool!.
%       
%       Die Haskell Suchmaschine Hoogle kann Bibliotheksfunktionen 
%       mithilfe des Typs schnell aufspüren;
%       tippen Sie also alternativ den Typ der gesuchten Funktion 
%       in Hoogle ein.
%       
%       Wenn Sie keinen Internetzugriff haben:
%       Bei der Installation der Haskell Plattform sollte 
%       die Dokumentation der Standardbibliotheken auch lokal installiert worden sein.
%       %
%       Eine weitere offline Möglichkeit zum Durchblättern der Dokumentation  noch der GHCI Befehl \verb|:browse|
%     \begin{loesung}
%       \begin{verbatim}
% Prelude> :browse Data.Bool
% (&&) :: Bool -> Bool -> Bool
% data Bool = False | True
% not :: Bool -> Bool
% otherwise :: Bool
% (||) :: Bool -> Bool -> Bool
% 
% Prelude> True && False 
% False
% Prelude> True || False 
% True
% Prelude> not True
% False \end{verbatim}
%     \end{loesung}
% 
%       
%       
% %   \item  Vergleiche?   
% % \end{enumerate}
% 
%   
% \end{aufgabe}
% 


\begin{aufgabe}{Auswertung}\label{Auswerten}%
Berechnen Sie möglichst mit Papier und Bleistift den Wert,
zu dem der gegebene Ausdruck
jeweils auswertet. Überprüfen Sie Ihr Ergebnis mit GHCI.

\begin{enumerate}
%   \item \verb|2 + 3 * 5== 35 `div` 2|
%       \begin{loesung}
%         \verb|True|
%       \end{loesung}
  \item \verb|'c':'o':'o':'l':"!"|
      \begin{loesung}
        \verb|"cool!"|
      \end{loesung}
  \item \verb|(4 == 5.0):[]|
      \begin{loesung}
        \verb|[False]|
      \end{loesung}
  \item \verb#let mond="käse" in if mond=="käse" && 1==2 then False else 1+1==2#
      \begin{loesung}
        \verb|True|
      \end{loesung}
  \item \begin{verbatim}
[ z | c <- "grotesk", c/='k', c/='r', c/='e' && c/='s'
    , let z = if c=='o' then 'u' else c ]\end{verbatim}
      \begin{loesung}
        \verb|"gut"|
      \end{loesung}
  \item \verb![(a,b) | a<-[1..5], b<-[5..1], a/=b]!
      \begin{loesung}
        \verb|[]|
       \emph{Zur Erinnerung:} Wenn man herunterzählen möchte, dann muss man die Schrittweite angeben.
       Die Liste \verb|[5..1]| ist leer, die Liste \verb|[5,4..1]| enthält dagegen 5 Zahlen.
      \end{loesung}
%   \item \verb![(y,z) | x<-['m'..'n'], y<-[1,200..500], let z = x:""]!
%       \begin{loesung}
%         \verb|[(1,"m"),(200,"m"),(399,"m"),(1,"n"),(200,"n"),(399,"n")]|
%       \end{loesung}
  \item \verb#(\x->"nope!") [c | c <- "yes!"]#
      \begin{loesung}
        \verb|"nope!"|
        Der erste Teil definiert eine anonyme Funktion, welches Ihr Argument ignoriert und immer den String \verb|"nope!"| zurückliefert.
        Der hintere Teil dieses Ausdrucks wertet wieder zu dem String \verb|"yes!| aus.
      \end{loesung}
\end{enumerate}
\end{aufgabe}


\begin{aufgabe}{Substitutionsmodell}\label{Substitutionsmodell1}
  %TODO: Lambda am anfang zu schwer!
  Werten Sie folgenden Ausdruck gemäß dem in der Vorlesung behandelten Substitutionsmodell
  aus:
    $$\Bigl(\bigl(\lam x \arr (\lam (y:\_) \arr x-y)\bigr)\ (3+4)\Bigr)\ (\fkt{tail}\ [1,2])$$
    
\begin{loesung}
  Wir verzichten hier gleich auf alle überschüssige Klammern gemäß der Klammerkonvention.
  
  Weiterhin rufen wir uns vorher noch 
  die Definition von \fkt{tail} aus Foliensatz~2 (oder auch aus der Standardbibliothek)
  in Erinnerung: $\fkt(tail)\ (\_:t) = t$.
  %
  Damit man besser verstehen kann, wie das Pattern-Matching bei der Funktionsanwendung
  von $\fkt{tail}$ und der anonymen Funktion abläuft,
  schreiben wir anstatt $[1,2]$ gleich $(1:(2:[\,]))$,
  beide Schreibweisen sind ja äquivalent.
  
 \begin{align*}
                & (\lam x \arr (\lam (y:\_) \arr x-y))\ \underline{(3+4)}\ (\fkt{tail}\ (1:(2:[\,]))) 
    \\\leadsto~ & (\lam x \arr (\lam (y:\_) \arr x-y))\ 7\ \underline{(\fkt{tail}\ (1:(2:[\,])))}
    \\\leadsto~ & (\lam x \arr (\lam (y:\_) \arr x-y))\ 7\ (2:[\,])
    \\\leadsto~ & (\lam (y:\_) \arr 7-y))\ (2:[\,])
    \\\leadsto~ & 7-2 \leadsto 5
 \end{align*}
\end{loesung}
    
%     \ (\underline{\fkt{negate}\ 1})        
%     \leadsto \underline{(\lam x \arr 43+x)\ (-1)}
%     \leadsto \underline{43+(-1)} \leadsto 42$$
\end{aufgabe}


\begin{aufgabe}{Pattern-Matching}\label{Pattern-Matching}%
  Schreiben Sie eine Funktion \verb|myAnd|,\footnote{Zur Vermeindung von Namenkonflikten mit der Standardbibliothek nicht ``and'' verwenden} 
  welche den Typ \linebreak\verb|Bool -> (Bool -> Bool)| hat
  und die logische Operation ``und'' implementiert.
  %
  Zur Übung der Vorlesungsinhalte sollen Sie diese Aufgabe gleich drei Mal lösen,
  jeweils mit einer anderen Einschränkung:  
  \begin{enumerate}
    \item Verwenden Sie bei der Definition ausschließlich Pattern-Matching.
    \item Verwenden Sie bei der Definition ausschließlich Wächter (keine konstanten Patterns).
    \item Verwenden Sie weder Pattern-Matching noch Wächter.
  \end{enumerate}
  Natürlich dürfen Sie in keinem Fall die Operatoren aus der Standardbibliothek verwenden!
  
  \begin{loesung}
    \begin{enumerate}
      \item Da der Typ vorgegeben war, können wir diese Aufgabe rein mechanisch lösen,
      indem wir für jeden möglichen Input eine Zeile mit dem Ergebnis hinschreiben:
    \begin{verbatim}
  myAnd1 :: Bool -> Bool -> Bool
  myAnd1 False False  = False    
  myAnd1 False True   = False
  myAnd1 True  False  = False
  myAnd1 True  True   = True  
    \end{verbatim}
    Wer sich etwas Schreibarbeit sparen will, kann Fälle mit dem gleichen Ergebnis zusammenfassen.
    Für Haskell macht das keinen Unterschied, aber für uns ist es wohl leichter lesbar.
    %
    Der einzige Spezialfall \verb|True True| schreiben wir an erster Stelle, damit dieser zuerst geprüft wird.
    %   
    Da alle restlichen Fälle auf \verb|False| abgebildet werden, können wir 
    diese nun pauschal mit einem Wildcard Pattern abarbeiten, welche immer matched:    
    \begin{verbatim}
  myAnd2 True True  = True 
  myAnd2 _          = False
    \end{verbatim}
    Eine alternative Lösung mit einem Pattern Match weniger wäre:
    \begin{verbatim}
  myAnd3 True    y  = y
  myAnd3 _          = False
    \end{verbatim}
%          Diese Variante nennt man auch nicht-strikt im zweiten Argument,
%          da dieses nicht inspiziert wir: Der Ausdruck 
%          \verb|myAnd2 (True,waldläufer)| terminiert, aber 
%          der Ausdruck \verb|myAnd1 (True,waldläufer)| terminiert nicht
%          (wobei \verb|waldläufer x = waldläufer x|).
%          Striktheit wird in der Vorlesung erst später behandelt.
      \item 
      \begin{verbatim}
  myAnd4 x y  
    | x, y      = True
    | otherwise = False
      \end{verbatim}
%       Alternative mit \verb|case|-Ausdruck
%       \begin{verbatim}
%   myAnd5 x y = case (x,y) of 
%     (x,y) | x && y    -> True
%           | otherwise -> False
%         \end{verbatim}
%       
        \item
        \begin{verbatim}
  myAnd6 x y = if x then y else False 
    \end{verbatim}
\end{enumerate}
  \footnotetext{Test}
  \end{loesung}
\end{aufgabe}


\pbreak{3}{3}


\begin{hausaufgabe}[keine]{Typen}{0}\label{Typen}%
Welchen Typ haben folgenden Ausdrücke?
\begin{enumerate}
  \item \verb|['a','b','c']|
    \begin{loesung}
      \verb|String| oder \verb|[Char]| (Typsynonym)
    \end{loesung}
  \item \verb|('a','b','c')|
    \begin{loesung}
      \verb|(Char,Char,Char)|
    \end{loesung}
  \item \verb|[(False,[1]),(True,[(2.0)])]|
    \begin{loesung}
      \verb|[(Bool, [Double])]|
    \end{loesung}
  \item \verb|([True,True],('z','o','o'))|
    \begin{loesung}
      \verb|([Bool], (Char, Char, Char))|
    \end{loesung}
  \item \verb|(\x -> ('a':x,False,([x])))|
    \begin{loesung}
      \verb|[Char] -> ([Char], Bool, [[Char]])|
    \end{loesung}
%   \item \verb@[z == 2.0 | x <- [1..10], let y = (2*x,x+1), let (z,_) = y]@
%       \begin{loesung}
%         \verb|[Bool]|
%       \end{loesung}
  \item \verb@[(\x y-> (x*2,y-1)) m n | m <- [1..5], even m, n <- [6..10]]@
      \begin{loesung}
        \verb|[(Integer, Integer)]|
      \end{loesung}
%   \item \verb|(\x -> (x,x))|
%       \begin{loesung}
%         \verb|a -> (a, a)|
%       \end{loesung}
%   \item \verb|(\(x,y) -> ([y],[],x:x:x:[]))|
%       \begin{loesung}
%         \verb|(a1, t) -> ([t], [a], [a1])|
%       \end{loesung}
%   \item \verb|(\a b c -> a:b:c:[])|
%       \begin{loesung}
%         \verb|a -> a -> a -> [a]|
%       \end{loesung}
%   \item \verb|[tail,init,reverse]|
%     \begin{loesung}
%       \verb|[[a]->[a]]|
%     \end{loesung}
\end{enumerate}
\emph{Hinweis:}
Kontrollieren Sie Ihre Antworten anschließend selbst mit GHCI!
Auch wenn wir Typinferenz in der Vorlesung noch nicht
behandelt haben, sollten Sie in der Lage sein,
die meisten dieser einfachen Aufgaben bereits
alleine mit Papier und Bleistift lösen zu können.
Dies Aufgabe ist eigentlich auch einfacher als Aufgabe~\ref{Auswerten}!
\end{hausaufgabe}

\pbreak{3}{3}

\begin{hausaufgabe}[PDF]{Substitutionsmodell II}{3}\label{Substitutionsmodell2}%
  Gegeben sind folgende zwei Funktionsdefinitionen
  \begin{verbatim}
    const x y =  x
    negate x  = -x
  \end{verbatim}
  \vspace*{-3EX}
  und ein Ausdruck
  $\fkt{const}\ \fkt{const}\ (\fkt{negate}\ 1)\ (\fkt{negate}\ 2)\ 3$
 \begin{enumerate}
  \item In dem gegebenen Ausdruck müssen wir die implizite Klammerkonvention berücksichtigen,
        siehe Folie~01-46.
        Geben Sie den Ausdruck noch einmal mit vollständiger, expliziter Klammerung an!
         
        \Loesung{
        Funktionsanwendung ist implizit links geklammert, d.h.\ 
        $$\Bigl(\bigl((\fkt{const}\ \fkt{const})\ (\fkt{negate}\ 1)\bigr)\ (\fkt{negate}\ 2)\Bigr)\ 3$$  }
        
  \item Werten Sie nun den Ausdruck schrittweise gemäß dem in der Vorlesung behandelten Substitutionsmodell
        vollständig aus; unterstreichen Sie jeweils den bearbeiteten Teilausdruck.
  
      \begin{loesung}  
        Eine mögliche Auswertung:
          \begin{align*}
          & \underline{\fkt{const}\ \fkt{const}\ (\fkt{negate}\ 1)}\ (\fkt{negate}\ 2)\ 3
          ~\leadsto~ \underline{\fkt{const}\ (\fkt{negate}\ 2)\ 3}
          ~\leadsto~ \underline{\fkt{negate}\ 2}
          ~\leadsto~ -2
      %     \end{align*} 
          \intertext{Wir dürfen aber auch eine andere Reihenfolge wählen, z.B.:}
      %     \begin{align*}
          & \fkt{const}\ \fkt{const}\ (\underline{\fkt{negate}\ 1})\ (\fkt{negate}\ 2)\ 3
          ~\leadsto~ \fkt{const}\ \fkt{const}\ (-1)\ (\underline{\fkt{negate}\ 2})\ 3
          \\ &~\leadsto~ \underline{\fkt{const}\ \fkt{const}\ (-1)}\ (-2)\ 3
          ~\leadsto~ \underline{\fkt{const}\ (-2)\ 3}    
          ~\leadsto~ -2
          \end{align*} 
        \end{loesung}
 \end{enumerate}      
      
  \begin{bewertung}
      1 Punkt auf die korrekte Klammerung;
      2 Punkte auf eine vollständig korrekte Auswertung, dabei pro Fehler 1 Punkt abziehen (also hier: 1 Fehler = 1 Punkt),
      keine Trostpunkte.
      
      Bei fehlerhafte Klammerung im ersten Teil können unter Beachtung des Folgesfehler trotzdem 
      noch die vollen 2 Punkte auf dem zwetien Teil erzielt werden!
  \end{bewertung}
    
\end{hausaufgabe}

\pbreak{3}{3}

% \begin{aufgabe}{Definitionsbereich}
%   Geben Sie jeweils den Definitionsbereich der folgenden Funktionen an:
%   \begin{enumerate}
%     \item $f :\NN \rightarrow \NN,\quad  f(x) = x^2$
%       \Loesung{$\dom(f)=\NN$}
%     \item $g:\RR \rightarrow \RR,\quad  g(x) = \begin{cases} x^3 - 4x + 11 & \text{falls } x > 1 \\ x^2 + 10x - 7 & \text{falls } x < 1 \end{cases}$
%       \Loesung{$\dom(g)=\RR\setminus\{1\}$}
%     \item $\text{pos}:\RR \rightarrow \RR,\quad \text{pos}(x) = \begin{cases} x & \text{falls } x > 0 \\ 0 & \text{sonst} \end{cases}$
%       \Loesung{$\dom(\text{pos})=\RR$}
%     \item $h:\RR \rightarrow \RR,\quad  h(x) = \frac{1}{pos(x^2-4)}$
%       \Loesung{$\dom(h)= \{x\in\RR \mid x<-2 \text{ oder } x>2 \}$}
%     \item $a:\RR\times\RR \rightarrow \RR,\quad  a(x,y) = \frac{y^2 }{x+1}$
%       \Loesung{$\dom(a)= (\RR\setminus\{-1\})\times\RR $}
%     \item $b:\RR\times\RR \rightarrow \RR \times \RR,\quad  b(x,y) = \begin{cases} (g(y),a(y,x)) & \text{falls } y \geq -1 \\ (-16,1) & \text{sonst} \end{cases}$
%       \Loesung{$\dom(b)= \RR \times (\RR\setminus\{-1,1\}) $}
%   \end{enumerate}
% \end{aufgabe}
%  
%\pbreak{3}{3}

\begin{hausaufgabe}[hss]{Modellieren mathematischer Funktion}{3}\label{mathefun}%
  Eine Ellipse um einen Mittelpunkt~$M$ 
  ist gegeben durch ihre große Halbachse~$a$
  und ihre kleine Halbachse~$b$.
  Auf der großen Achse 
  liegen die beiden Brennpunkte~$F_1,F_2$ der Ellipse.
  Den Abstand~$e$ dieser Brennpunkte vom Mittelpunkt
  nennt man die Exzentrizität der Ellipse.

  Die Exzentrizität hat einen Wert zwischen $0$ und $a$ 
  und ist ein Maß dafür,
  wie "`länglich"' die Ellipse ist.
  Die Exzentrizität hat den Wert $0$ im Fall $a=b$,
  wenn also die Ellipse zu einem Kreis mit Radius~$a$ entartet.
  Die  Exzentrizität hat den Wert $a$ im Fall $b=0$,
  wenn also die Ellipse zu einer Strecke der Länge~$2a$ entartet.

  \vspace*{1ex}
  \begin{minipage}{30ex}
  \begin{center}
 \includegraphics[width=10EM,keepaspectratio=true]{./ellipse.png}
 % ellipse.png: 0x0 pixel, 300dpi, 0.00x0.00 cm, bb=
\end{center}
 \end{minipage}
  \hspace*{\fill}
  \begin{tabular}{ll}
    \multicolumn{2}{l}{Ellipse mit Halbachsen $a$ und $b$ mit $a \ge b \ge 0$}\\\hline
    Flächeninhalt: & $\pi a b$ \\
    Exzentrizität: & $\sqrt{a^2 - b^2}$ \\
    Umfang:        & $\pi \Big( \frac{3}{2}(a+b) - \sqrt{a b} \Big)$
  \end{tabular}
  \hspace*{\fill}
  \vspace*{3ex}

  In der obigen Tabelle sind die Formeln zur Berechnung
  von Flächeninhalt, Exzentrizität und Umfang angegeben.
%   (Die Formel für den Umfang liefert nur einen
%   --- allerdings ziemlich genauen ---
%   Näherungswert,
%   weil der exakte mathematische Wert
%   durch eine unendliche Reihe gegeben ist,
%   also als Grenzwert einer Summe mit unendlich vielen Summanden.)

  Schreiben Sie eine Datei \texttt{H1-3.hs}, welche diese Funktionen implementiert:
  \begin{verbatim}
   flaecheninhalt :: Double -> Double -> Double
   exzentrizitaet :: Double -> Double -> Double
   umfang         :: Double -> Double -> Double  \end{verbatim}
  
  \paragraph{Hinweise}
  \begin{itemize}
    \item Zur Vereinfachung müssen Sie nicht überprüfen,
          dass die Bedingung $a \ge b \ge 0$ tatsächlich eingehalten wird.
    \item Folgende Funktionen aus der Standardbibliothek dürfen Sie zusätzlich 
      zu den üblichen mathematischen Operationen \verb|+|,\verb|-|,\verb|*|,\verb|/|,\verb|^|
    verwenden: \verb|pi|, \verb|sqrt|. Schlagen Sie diese Funktionen ggf.\ in der Dokumentation der 
    Standardbibliothek nach.
    \item \emph{Beispiel}          
  Eine Funktion zur Berechnung des Flächeninhalts eines Kreises, könnte man in Haskell 
  wie folgt definieren: \begin{verbatim}  
  kreisfläche :: Double -> Double 
  kreisfläche r = pi * r * r \end{verbatim}   
    
    \item Punktabzug, falls Funktionsnamen, Signatur oder Dateiname nicht wie angegeben sind.
  \end{itemize}

  \begin{loesung}
    \verbatiminput{ellipse.hs}
     \begin{bewertung}
      3 Punkte wenn alle automatischen Tests bestanden sind und alles korrekt ist.
      2 Punkte bei 1--2 Fehlern oder auch schon falls nicht alle Tests bestanden werden (z.B. bei Mißachtung des Abgabeformats).
      1 Trostpunkt für einen erkennbar bemühten Lösungsansatz.
    \end{bewertung}
  \end{loesung}

\end{hausaufgabe}



\vspace{\fill}
\paragraph{Abgabe:}
Lösungen zu den Hausaufgaben können bis Dienstag, den 28.04.2015, 11:00 Uhr mit \href{https://uniworx.ifi.lmu.de/?action=uniworxCourseWelcome&id=398}{UniworX} abgegeben werden.

Aufgrund des Klausurbonus müssen die Hausaufgaben von Ihnen alleine gelöst werden.
Abschreiben bei den Hausaufgaben gilt als Betrug und kann zum Ausschluss von der Klausur zur Vorlesung führen.
% Achten Sie besonders auf saubere Beweise!
% Es ist ratsam, die Beweise so ausführlich wie möglich zu führen.
% \emph{Zur Erinnerung:} Die Hausaufgaben müssen von Ihnen alleine gelöst werden; eine Abgabe in Gruppen ist nicht zulässig!
% Gruppenabgaben sind nicht zulässig.
\end{document}
